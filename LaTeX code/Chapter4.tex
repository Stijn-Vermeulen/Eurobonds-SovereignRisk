\pagestyle{headings}
\chapter*{Conclusion}\addcontentsline{toc}{chapter}{Conclusion}
The aim of this dissertation was to study the impact of Eurobonds in a setting of sovereign default following the set-up of Hatchondo et al. (2017). The main contribution is the simultaneous analysis of an introduction of a non-defaultable asset and a debt restructuring process based on a Nash-Bargain game. If the introduction of a Eurobond is considered in isolation, the welfare gains are situated in the short-run, while in the long-run defaultable debt levels and risk premia converge to their original level. When a bargain game is introduced à la Yue (2010), the long-run aspect changes. Investors will have an incentive to re-evaluate their bargain position and will demand a higher recovery rate after a default. The higher recovery rate in turn leads to price and debt adjustments by both investors and the government, finally resulting in a higher overall debt level. In our model, Eurobonds alone do not provide incentives to reduce debt levels. This supports the argument that if the purpose of Eurobonds is to strengthen fiscal discipline, they should be introduced with additional fiscal rules.\\

However, our results do warrant several remarks which can be used as foundation for future research. First of all, a first deviation we made from Hatchondo et al. (2017) is to only consider the case of one period defaultable debt. If long-term debt is introduced, interest rates are no longer only dependent on next period's default risk, but also on subsequent periods. In their paper, they find that the introduction of Eurobonds does reduce the risk premium, however not completely to zero. Long-term debt could therefore lead to a more nuanced conclusion about the distribution of bargaining power, since the decrease in default risk is what triggers investors to re-evaluate their bargain strategy. For a complete analysis, the case of long-term debt should also be considered.\\

A second deviation from the original work is made by choosing a Discrete State Space approach to solve the model. In the original paper, Hatchondo et al. used linear interpolation for endowment levels and spline interpolation for defaultable debt levels. The advantage of using interpolation methods is the potential gains in efficiency and accuracy as discussed in Hatchondo et al. (2010). The analysis from the robustness of the grid specification and multiple equilibria gives us reason to be cautious about the results. Therefore, further research should be performed to test the validity of our model and interpolation methods could be a powerful tool to do so.\\

A final remark concerns the introduction of a credible debt rule. We did not cover how this rule could be implemented and different rules might have different results. A complementary issue is that the assumption of an exogenous output process rules out the influence that government borrowing might have on growth or the business cycle. Further research in this area can prove fruitful in our understanding of Eurobonds.\\ 
