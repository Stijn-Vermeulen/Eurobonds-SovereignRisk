
\pagestyle{headings}

\chapter{Literature on Eurobonds \& sovereign default}
To better our understanding of "Eurobonds", this dissertation will rely and contribute to different strands of literature. Exploring the theory and concepts behind it branches out in economic, political and financial theory. Out of this wide range of literature two particular strands will be of main interest to us: the literature on debt mutualisation or creation of safe assets and the literature dealing with sovereign debt and default models. In section 1.1, the former strand of literature will be reviewed and the advantages and disadvantages will be discussed of different proposals made over the years. In section 1.2 we will continue to discuss the second strand of literature and elaborate on the contributions that make the analysis of "Eurobonds" possible in the context of a default model.\\


\section{The literature on "Eurobonds"}
In the aftermath of the financial crisis and during the sovereign debt crisis several voices rose up to advocate the issuance of common debt as a crisis resolution tool (Claessens, Mody \& Vallée, 2012). Nonetheless, the idea of issuing common debt for member states is not novel and can be traced as far back to 1974 with the Haferkamp's "Community loans" (Haferkamp, 1974) and the "Union Bonds" idea by Jacques Delors (European Commission, 1993). More recent proposals by, for example, Moesen \& De Grauwe (2009), Delpla \& Von Weizsäcker (2010), Juncker \& Tremonti (2010) and Philippon \& Hellwig (2011) weighed in with their suggestions on how to implement and organize the creation of such a "Eurobond". In even more recent times, calls have been made to issue common debt specifically to help governments in their combat to the adverse economic effects of the Covid-19 pandemic (e.g. Bénassy-Quéré et al., 2020a; Bénassy-Quéré et al., 2020b; Botta et al., 2020).\\ 

\subsection{Potential benefits of "Eurobonds"}
Despite the differences in implementation and organization, most proposals share one or more common objective. Claessens et al. (2012) summarize these objectives into three broad categories: improvement of fiscal risk-sharing \& discipline, improvement of financial stability and facilitating the monetary policy transmission.\\
\clearpage
The current fiscal framework in the European Union is embodied by the Stability and Growth Pact (European Commission, 2019). Some, however, advocate further fiscal integration (e.g. Juncker et al., 2015) and "Eurobonds" could be a possible first step towards a full-fledged fiscal union with strong risk-sharing and discipline mechanisms (Claessens et al., 2012; De La Dehesa, 2011). However, there is no agreement on whether common issuance of debt alone would increase or decrease fiscal discipline (Wyplosz, 2011; Delpla \& Von Weizsäcker, 2010). Some authors propose that imposing additional fiscal rules and strong economic governance could help in ensuring fiscal discipline is maintained after the introduction of "Eurobonds" (Delpla \& Von Weizsäcker, 2010; Eijffinger, 2011). \\

A second area in which "Eurobonds" could improve welfare is the one of financial stability. The "diabolical loop" between banks and sovereigns is often attributed to be one key contributor to the sovereign debt crisis. As banks had a home bias in their sovereign debt holdings, they were more vulnerable to negative fiscal positions of their domestic state. Vice versa, if banks reduce their lending to compensate for losses on sovereign debt holdings, the state itself suffers as well due to decreasing economic activity (Brunnermeier et al., 2016; Collignon, 2011). A "Eurobond" could therefore become a safe common asset which could weaken this link between bank and state. Besides weakening the diabolical loop, financial stability can also be strengthened by weakening the risks of large flight-to-quality streams and improving the liquidity of the European bond market (Claessens et al., 2012; De La Dehesa, 2011; Delpla \& Von Weizsäcker, 2010).\\

Lastly, Claessens et al. (2012) also argue that "Eurobonds" could improve the monetary transmission mechanism by providing a unified market for European sovereign debt on which the European Central Bank could intervene.\\
\subsection{Objections to "Eurobonds"}
Notwithstanding the potential benefits of "Eurobonds", critics have pointed out the potential drawbacks of mutualising debt. Two common and related arguments, in particular, have been brought up by the (political) opposition: the problem of moral hazard and potential conflict with the "no-bailout" clause. The problem of moral hazard could occur if all member states would be liable for each other's debt. This shared liability might induce fiscally weaker states to borrow more as they can partially pass on their sovereign risk to others. Introducing "Eurobonds" could thus, for example, lead to a situation where implicit transfers are made from the northern, fiscally strong states to the southern, fiscally weaker states (Claessens et al., 2012; De La Dehesa, 2011; Wyplosz, 2011).\\

A second concern is the conflicting nature between "Eurobonds" and the "no-bailout" clause. The possibility of member states passing on risk to each other was recognized with the formation of the monetary union. Thus, a clause was included in the Treaty on the Functioning of the European Union (2007) that explicitly states no state shall be liable for another's debt commitments. This could present judicial difficulties for implementing "Eurobonds", as many proposals involve member states to jointly guarantee their repayment. Implementing jointly guaranteed "Eurobonds" under the "no-bailout" clause would either require a work-around past the clause or a change in the international treaty.\\
\subsection{Defining Eurobonds}
With the wide variety in proposals and similar variety in names they have been given over the years, e.g. Redemption Bonds, Eurobills, Blue-Red Bonds, coronabonds, ESBies, Sovereign Bond Backed Securities, etc., it should come as no surprise that \textit{the} Eurobond proposal does not truly exist. Therefore, to keep the analysis tractable this dissertation will use \textit{Eurobond} or \textit{non-defaultable debt} as an umbrella term for a specific category of common debt issuance, namely those that involve \textit{joint and several} guarantees. With joint and several guarantees each individual member state is responsible for full repayment of the mutualised debt, even if another member cannot repay (Claessens et al., 2012).  If this arrangement is considered to be credible, than from the investors' point of view this form of debt is non-defaultable, i.e. investors will always be repaid, unless all member states unanimously decide to default. A drawback of only considering the proposals that involve joint and several guarantees is that it leaves out proposals that just involve pooling and/or tranching of sovereign debt. The most notable proposals within this category are the ESBies (Brunnermeier et al., 2016 ) and Sovereign Bond Backed Securities (ESRB, 2018). The possible impact of these types of assets in a sovereign default setting lies out of the scope of this paper and is left for future research.\\

Even by only focusing on proposals involving joint and several guarantees, a lot of different forms of debt mutualisation remain. From this extensive list three notable proposals would match to our analysis: The Eurobills (Philippon \& Hellwig, 2011), the Blue-Red bonds (Delpla \& Von Weizsäcker, 2010) and the Redemption Pact\footnote{The goal of the Redemption Pact is to reduce member state's debt over 60\% of GDP and this within 20-25 years. The original proposal is thus only temporary in nature. To keep our analysis tractable and streamlined, however, we will assume Redemption Bonds to be a permanent form of debt.} (Doluca et al., 2012). With Eurobills, Philippon \& Hellwig propose to issue short-term debt up to 10\% of GDP. They argue that by issuing short-maturity debt a safe asset is created, while still maintaining some discipline as it needs to be paid back in the short-term. Delpla \& Von Weizsäcker and Doluca et al., on the other hand, propose to both tranche and pool debt in a common, jointly guaranteed pool and a risky pool. Both proposals are close to each other's inverse, as the Blue bond would be a safe asset jointly guaranteed up to 60\% of GDP, while the safe Redemption bonds would be for any existing debt above the target of 60\% of GDP. If Redemption bonds would have been introduced for the Euro Area in 2019, safe debt would amount to 24.1\% of GDP (Eurostat, 2020b).\\

Throughout the rest of this dissertation, the Eurobill proposal will be used as a benchmark. Choosing to focus on Eurobills has its advantages for the analysis in chapter 2 and 3. First of all, the model in chapter 2 is based on the work by Hatchondo et al. (2017). In their work the non-defaultable debt limit is set at 10\% and is considered to be of a short-term nature. Setting our limit also to 10\% allows us to draw comparisons between their paper and our results. A second reason to focus on the Eurobill proposal is that Claessens et al. (2012) suggest this to be one possible first step which other proposals can follow, e.g. the Blue-Red bond proposal with a higher limit on non-defaultable debt. This study would, therefore, research the effects of a first, smaller step towards a possible full-fledged fiscal union. The third and last reason is that by setting the limit of mutualised debt low, the problem of moral hazard is minimised (Philippon \& Hellwig, 2011). Later in the robustness section we will check whether the conclusion changes for other proposals with higher limits on Eurobonds.\\

\section{Related sovereign debt \& default literature}
The framework outlined in chapter 2 will be based on the seminal work by Eaton \& Gersovitz (1981) and features a small open economy with a government borrowing non-state contingent debt from international lenders. Recent literature has extended this framework and calibrated a wide range of models, mostly so to explain the default episodes observed in emerging economies, e.g. Argentina (Aguiar \& Gopinath, 2006; Arellano, 2008). Some of these additions to the basic Eaton-Gersovitz model range from the introduction of long-term debt (Arellano \& Ramanarayanan, 2012; Chatterjee \& Eyigungor, 2012; Hatchondo \& Martinez, 2009) to endogenising default costs (Mendoza \& Yue, 2012) or introducing political uncertainty (Cuadra \& Sapriza, 2008; Önder \& Sunel, 2020). Two papers out of the sovereign default literature will be of most interest for this dissertation: Hatchondo, Martinez \& Önder (2017) and Yue (2010).\\

Hatchondo et al. (2017) extend the standard Eaton-Gersovitz setting by introducing a non-defaultable asset to which the government still has access to even after a default. The asset is considered to be risk-free and therefore resemblant of a Eurobond with joint and several guarantees. Chapter 2 will follow their set-up and also introduce a non-defaultable asset. It will deviate from Hatchondo et al. by assuming that all debt is short-term. An incomplete list of other related literature that introduce a third asset in the government's problem are Fink \& Scholl (2016), Boz (2011) and Alfaro \& Kanczuk (2007). Alfaro \& Kanczuk introduce international reserves into the government's problem and show that optimal policy is not to accumulate reserves. Fink \& Scholl and Boz, on the other hand, develop a model with conditional loans from international financial institutions (e.g. the IMF). These loans are strictly enforceable and thus non-defaultable. The difference with Hatchondo et al.'s non-defaultable asset, however, comes from the conditionality on fiscal policy imposed with IFI loans. Fink \& Scholl's results show that the possibility of IFI loans increases long-run debt levels due to lower borrowing costs. This dissertation will find a similar result, rather it is caused by the second key feature in our model: renegotiation after default and bargaining power.\\

The second key feature in our model will be a renegotiation scheme allowing for a partial, rather than full-on sovereign default, as in the work by Yue (2010). Yue added a Nash-bargain process to an Eaton-Gersovitz model where after a default a government needs to renegotiate with investors. Bargain power is distributed between the sovereign and lenders and if lenders' level of bargain power is non-zero, then the recovery rate will be non-zero as well. The non-zero recovery rate will allow for higher sustainable levels of debt due to investors taking the recovery into account when they price debt. The analysis in chapter 3 will show that bargaining power could be an important factor to consider when introducing Eurobonds.\\