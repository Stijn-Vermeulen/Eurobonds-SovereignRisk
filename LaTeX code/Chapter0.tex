\mainmatter
\pagestyle{headings}

\chapter*{Introduction}\addcontentsline{toc}{chapter}{Introduction}
Shortly after the 2008 Financial crisis the Eurozone was hit by a second, sovereign debt crisis during which vulnerabilities of the monetary union were laid bare. This crisis was characterized by a "diabolical loop" between sovereign states and their domestic banking system. As banks were obligated to meet capital requirements, sovereign bonds were an attractive asset because of the zero risk-weight given by regulators. These requirements in combination with a home bias led to a loop between sovereign risk and bank risk where the deterioration in one could lead to deterioration in the other. Additionally, while in the years preceding the Financial crisis capital flowed from non-vulnerable to vulnerable by the prospect of higher yields, the flows reversed around 2009 when investors sought safety. The flight of capital drove down the borrowing costs of non-vulnerable member states and increased the costs for vulnerable, more risky states. The result was a deep recession and large increases in interest rates for member states in distress, e.g. Greece, Portugal and Ireland (Brunnermeier et al., 2016; Lane, 2012).\\

Both during and after the years of the crises, different proposals have been made to improve the Union's fiscal sustainability and increase stability of the financial system to help the Eurozone withstand the next crisis (e.g. Juncker et al., 2015; Brunnermeier et al., 2016). In this dissertation we will discuss the implications of one such proposal: "Eurobonds". "Eurobonds" have been proposed to be a safe form of debt, possibly mutualised and guaranteed by all member states which could benefit the European integration. To study its possible impact, we will work with the framework of Hatchondo, Martinez \& Önder (2017) and implement a safe, non-defaultable asset in an Eaton-Gersovitz setting. In particular, we pose the question as to what the effects of introducing "Eurobonds" are on default incentives and overall welfare. Our conclusion will be that this depends on the relative bargain power distribution between member states and investors.\\

This dissertation will be organized in three main chapters. Chapter 1 will briefly discuss the underlying ideas and goals of different "Eurobond" proposals and relate these ideas to the literature on sovereign debt and default models. Chapter 2 will continue by outlining the construction of the sovereign default model which will be used to simulate the potential introduction of a non-defaultable asset (i.e. a "Eurobond"). In chapter 3 we will analyse the results and perform robustness checks of our model and see what the welfare implications of "Eurobonds" are. Finally, we will conclude this dissertation by discussing the policy implications of our results and possible avenues of further research.\\


