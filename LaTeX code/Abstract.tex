\chapter*{Abstract\hfill} \addcontentsline{toc}{chapter}{Abstract}

This dissertation will research the impact of introducing a "Eurobond" in a sovereign default model à la Eaton-Gersovitz (1981). Through partially replicating the results of Hatchondo, Martinez \& Önder (2017) we find that for one-period debt introducing a non-defaultable asset leads to a temporary increase in consumption and temporary decrease in the risk premium. These benefits are, however, short lived as the economy normalizes after 4 quarters. Additionally, this dissertation extends Hatchondo et al. (2017)'s original work by implementing a Nash-Bargain game for debt restructuring as performed by Yue (2010). We will show that assuming a bargain game for debt restructuring changes the incentives for borrowing and in combination with "Eurobonds" will lead to higher overall debt levels. The findings of the dissertation complements earlier work on the introduction of non-defaultable debt and highlights the relevance of the distribution of power between sovereign states and investors in the "Eurobond" policy debate.

